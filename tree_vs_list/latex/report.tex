\documentclass[a4paper,11pt]{article}

\usepackage[utf8]{inputenc}

\usepackage{minted}

\begin{document}

\title{
    \textbf{Tree Vs List}
}
\author{Malcolm Liljedahl}
\date{Spring Term 2022}

\maketitle

\section*{Introduction}

For this assignment we were supposed to create an ordered list and an ordered tree and then compare the insertion times between these structures.

\section{List}

So for the list implementation, 2 functions were used. The first one called $list\_new$ simply creates an empty list. The other one called $list\_insert$ is a recursive function that's divided up into three smaller parts. The first one inserts the argument(element) if the list is empty, the second one inserts the argument(element) at the beginning of the list if the argument is smaller than the current head, because then we know that this element is the smallest one. The third part is run if the first two previous parts isn't run. This part recursively compare the element we want to insert with the elements in the last. It goes from head to tail, and compares until it finds it's spot and then completes the list again.

\begin{minted}{elixir}
 def list_new() do [] end

 def list_insert(e, []) do [e] end
 def list_insert(e, [h|t]) when e <= h do [e,h|t] end
 def list_insert(e, [h|t]) do [h|list_insert(e, t)] end
\end{minted}


\section{Tree}

For the tree implementation also 2 functions were used. The first one works similarly to the list one and is called $tree\_new$. The second function is split into multiple parts. The first three parts can be seen as base cases. Which are called upon if the tree is empty, or if there's only one current node(head) in the tree, then we will insert the new node(e) into the tree and make the current node(head) in the tree a "leaf" to the new node(e) or IF the new node(e) is smaller then the current node(head)
If the tree is of a bigger scale than 0 or 1 nodes, then the two later parts will be executed, here we recursively move through the tree and compare our new node(e) with the rest of the nodes in the tree. Starting with the root, until the right place for the new node(e) is found.

\begin{minted}{elixir}
def tree_new() do :nil end

def tree_insert(e, :nil) do {:leaf, e} end

def tree_insert(e, {:leaf, head}=right) when e < head do 
    {:node, e, :nil, right} end

def tree_insert(e, {:leaf, _}=left)  do 
    {:node, e, left, :nil} end
    
def tree_insert(e, {:node, head, left, right}) when e < head do 
    {:node, head, tree_insert(e, left), right} end
    
def tree_insert(e, {:node, head, left, right}) do
    {:node, head, left, tree_insert(e, right)} end
\end{minted}

\section{Bench}

Bench is a function that was used for the test it calls itself 100 times via a loop. The list ls determines the length of the list/tree that are run. All elements in ls are thrown into the bench function together with four other functions. Two that creates a tree and a list and two that inserts the elements into these data structures. And then together with a function called time the list of various size with random elements are inserted in the created list and tree. Timer calculates how long it takes to create the list/tree.

\begin{minted}{elixir}
def bench() do bench(100) end

def bench(l) do

    ls = [16,32,64,128,256,512,1024,2*1024,4*1024,8*1024]
    
    time = fn (i, f) ->
      seq = Enum.map(1..i, fn(_) -> :rand.uniform(100000) end)
      elem(:timer.tc(fn () -> loop(l, fn -> f.(seq) end) end),0)
    end

    bench = fn (i) ->

      list = fn (seq) ->
        List.foldr(seq, list_new(), fn (e, acc) -> list_insert(e, acc) end)
      end
    
      tree = fn (seq) ->
        List.foldr(seq, tree_new(), fn (e, acc) -> tree_insert(e, acc) end)
      end
    
      tl = time.(i, list)
      tt = time.(i, tree)
    
      IO.write("  #{tl}\t\t\t#{tt}\n")
    end

    IO.write("# benchmark of lists and tree (loop: #{l}) \n")
    Enum.map(ls, bench)

    :ok
end

def loop(0,_) do :ok end
    def loop(n, f) do
    f.()
    loop(n-1, f)
end
\end{minted}

\section{Result}

The elements used for the tests are randomized, so there's a new result every time the program is run. Though the result are pretty similar, I chose the display one of the more arbitrary tests.

Here are the following results:

\begin{table}
\begin{center}
\begin{tabular}{|c|c|}
\textbf{list} & \textbf{tree}\\
\hline
       102 &     0\\
      204 &     102\\
        921 &     409\\
       3686 &     819\\
      13414 &     2457\\
        61952 &     7782\\
      256512 &     20377\\
      1183539 &     53862\\
      4457369 &     121753\\
        21034700 &     354201\\
\end{tabular}
\caption{Insertion time of elements in a list and a tree, run time in microseconds}
\label{tab:table1}
\end{center}
\end{table}

From this table we can see that the tree data structure used is at first twice as fast as the lists and then the lists time increases exponentially. So overall the tree data structure is the right choice if run time is the main focus.




\end{document}
