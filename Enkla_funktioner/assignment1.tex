\documentclass[a4paper,11pt]{article}
\usepackage[utf8]{inputenc}
\usepackage{minted}
\usepackage{graphicx}

\begin{document}

\title{
    \textbf{Enkla Funktioner}
}
\author{Malcolm Liljedahl}
\date{Spring Term 2022}

\maketitle

\section*{Introduction}

For this assignment two functions were created. The first one is the first elixir function I wrote without following any norms/standards hence the use of if/else statements. The second function is more elixir specific via the case block and the use of recursion.

The first function called sum takes in an element and a list as parameters and if the element is not currently in the list and it's added via pattern matching. But if the element is in the list then a statement is printed to the console instead.

The second function is a fibonnaci function. It takes in a number from the user user and outputs the fibonnaci summary to the console. Here the "case" is used instead of and if/else block. So if the number is 0 then then 0 is printed to the console, if the number is 1 then then 1 is printed to the console. But if it's a number which is not 0 nor 1 then the fibonnaci number is being calculated recursively via fib(n-1) + fib(n-2).

\section*{Sections}

The first function was pretty straightforward because I've used if/else blocks a lot in all the other courses. And we got a taste of pattern matching in both discrete math and in JavaScript course.

The second function was a bit more tricky though, especially when it was supposed to be recursive. The case block wasn't to hard to understand but how to make it work properly as a recursive function was the problem. I managed to figure out how to do it recursively but it was via a lot of if/else blocks, until I saw the course video which show how to use a case block instead.


\section*{Inserting code}

\begin{minted}{elixir}

#First function:


  # return the sum of all elements in the list l, assume that all elements are integers
  def sum2([]) do
    0
  end

  def sum2([head|tails]) do
    head + sum2(tails)
  end
  
 

#Second function:
    
    # fibonnaci function written recursively
    def fib(n) do
        case n do
            0 -> 0
            1 -> 1
            _ -> fib(n-1) + fib(n-2)
        end
      end
  
 \end{minted}
 
 This is the list used for the first function:

\includegraphics[scale=0.5]{images/enkla_funktioner_1.jpg}

And the output was:

\includegraphics[scale=0.5]{images/enkla_funktioner_2.png}

And after running the fibonnaci function the console looked like this:

\includegraphics[scale=0.5]{images/enkla_funktioner_3.png}

\end{document}