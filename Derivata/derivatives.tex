d\documentclass[a4paper,11pt]{article}

\usepackage[utf8]{inputenc}

\usepackage{minted}

\begin{document}

\title{
    \textbf{Derivative}
}
\author{Malcolm Liljedahl}
\date{Spring Term 2022}

\maketitle

\section*{Introduction}

For this assignment we were supposed to create function that can via help of pattern matching take the derivative of mathematical function and display it for the user in various ways.

\section{Representing functions}

In Elixir it is possible to represent a function as  $f = fn(x) -> x * 2 $ by picking apart the function and representing it as a multiplication between the variable(argument) and a constant(2).

\section{Math expressions}

To make it easier and "cleaner" to write the code the first step was to create "type variables" that we could reuse. In other words, we create tuples for all expressions/variables.

\begin{minted}{elixir}
  @type literal() :: {:num, number()} | {:var, atom()}
  @type expr() :: literal()
  | {:add, expr(), expr()}
  | {:mul, expr(), expr()}
  | {:expr, expr(), literal()}
  | {:ln, expr()}
  | {:sin, expr()}
  | {:cos, expr() }
\end{minted}

\section{Derivatives}
Then we could implement the rules that we will use to calculate the derivatives. The implementation is pretty basic as long as you understand thoroughly how the derivation rules work.

Rules were implemented for addition, multiplication, exponents, sine, cosine, natural logarithms and square root. 

A more ideal solution to this would be to implement some fraction pattern, but instead to make my life easier I wrote the fractions as exponents instead. For instance $1/x$ is written as $x^-1$

\begin{minted}{elixir}
def deriv({:num, _}, _) do {:num, 0} end
def deriv({:var, v}, v) do {:num, 1} end
def deriv({:var, _}, _) do {:num, 0} end

def deriv({:add, e1, e2}, v) do 
    {:add, deriv(e1,v), deriv(e2,v)} end

def deriv({:mul, e1, e2}, v) do 
    {:add, {:mul, deriv(e1, v), e2}, {:mul, e1, deriv(e2, v)}} end

def deriv({:exp, e, {:num, n}}, v) do 
    {:mul, {:mul, {:num, n}, {:exp, e, {:num, n-1}}}, deriv(e, v)} end

def deriv({:exp, {:var, x}, {:var, n}}, v) do 
    {:mul, {:var, n}, {:exp, {:var, x}, {:add, {:var, n}, {:num, -1}}}} end #x^n

def deriv({:sin, e}, v) do 
    {:mul, deriv(e, v), {:cos, e}} end #sin(x)

def deriv({:cos, e}, v) do 
    {:mul, {:num, -1}, {:mul, deriv(e, v), {:sin, e}}} end #cos(x)

def deriv({:ln,{:var, x}}, _v) do 
    {:exp, {:var,x}, {:num, -1}} end #ln(x)
\end{minted}

\section{Printing the results}
Another problem is for the user to be able to easily read the output to the console. Some additional functions were created for this exact purpose. A lot of the later added derivative rules could use the earlier created pprint functions. So not a lot of new coded was needed when the foundation had been made.

\begin{minted}{elixir}
def pprint({:num, n}) do "#{n}" end
def pprint({:var, v}) do "#{v}" end
def pprint({:add, e1, e2}) do "(#{pprint(e1)} + #{pprint(e2)})" end
def pprint({:mul, e1, e2}) do "#{pprint(e1)} * #{pprint(e2)}" end
def pprint({:exp, e1, e2}) do "(#{pprint(e1)})^(#{pprint(e2)})" end
def pprint({:ln, e1}) do "ln(#{pprint(e1)})" end
def pprint({:sin, e1}) do "sin(#{pprint(e1)})" end
def pprint({:cos, e1}) do "cos(#{pprint(e1)})" end
\end{minted}

\section{Simplify the functions}
A lot of unnecessary literals were printed to the console for instance $x$ could be displayed as $x * 1 $, to avoid this some functions were created to simplify the output, so it looks more like if the math were calculated by hand.

\begin{minted}{elixir}
def simplify({:add, e1, e2}) do simplify_add(simplify(e1), simplify(e2)) end
def simplify({:mul, e1, e2}) do simplify_mul(simplify(e1), simplify(e2)) end
def simplify({:exp, e1, e2}) do simplify_exp(simplify(e1), simplify(e2)) end
def simplify(e) do e end

def simplify_add(e1, {:num, 0}) do e1 end
def simplify_add({:num, 0}, e2) do e2 end
def simplify_add({:num, n1}, {:num, n2}) do {:num, n1+n2} end
def simplify_add(e1, e2) do {:add, e1, e2} end

def simplify_mul({:num, 0}, _) do {:num, 0} end
def simplify_mul(_, {:num, 0}) do {:num, 0} end
def simplify_mul(e1, {:num, 1}) do e1 end
def simplify_mul({:num, 1}, e2) do e2 end
def simplify_mul({:num, n1}, {:num, n2}) do {:num, n1*n2} end
def simplify_mul(e1, e2) do {:mul, e1, e2} end

def simplify_exp(_, {:num, 0}) do {:num, 1} end
def simplify_exp(e1, {:num, 1}) do e1 end
def simplify_exp({:num, n1}, {:num, n2}) do {:num, :math.pow(n1,n2)} end
def simplify_exp(e1, e2) do {:exp, e1 ,e2} end
\end{minted}

\section{Conclusion}
Overall this was an interesting assignment which show that when you use recursion and pattern matching you save a lot of code space. You can do a lot with just a few lines code, and the program can easily be built upon if other features and functions are to be implemented.

\end{document}